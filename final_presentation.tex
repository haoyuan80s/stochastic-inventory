\documentclass{beamer}
\usepackage{stackengine}
\renewcommand\useanchorwidth{T}
\usepackage{xcolor}
\def\theyearwidth{1.5pt}
\def\mystrut{\rule{0ex}{.1ex}}
\def\myyrstrut{\rule[-1ex]{0ex}{2ex}}
\newlength\yrsfboxrule
\yrsfboxrule .4\fboxrule
\newcommand\yearwidth[1]{\def\theyearwidth{#1}\ignorespaces}
\newcommand\skipyears[2][black]{%
  \fboxrule\yrsfboxrule%
  \fboxsep=-\yrsfboxrule%
  \fcolorbox{#1}{#1}{\mystrut\hspace{#2}}%
  \ignorespaces%
}
\newcommand\showyear[2][black]{%
  \fboxsep=0pt%
  \stackunder[2pt]{%
    \colorbox{#1}{\myyrstrut\hspace{\theyearwidth}}%
  }{\tiny#2}%
  \ignorespaces%
}
\usetheme{Amsterdam}

\title{Approximation Algorithms for stochastic Inventory Control Models}
\subtitle{A periodic-review stochastic inventory control problem}
\author[Hao Yuan, Feng Wei, Blake Miller]{Hao Yuan, Feng Wei, Blake Miller \\ {\ttfamily Github:\href{https://github.com/blakeapm/stochastic-inventory}{github.com/blakeapm/stochastic-inventory}}}
\date{\today}

\begin{document}
  \frame{\titlepage}
  \section{The Periodic-Review Stochastic Inventory Control Problem}
  \subsection{General Concepts}
  \begin{frame}
    \frametitle{The Periodic-Review Stochastic Inventory Control Problem}
    \begin{itemize}
      \item Computing a provably efficient inventory control policies is difficult
      \item This is due to the nature of demand:
        \begin{itemize}
          \item correlated
          \item non-stationary (time-dependent)
          \item evolving
        \end{itemize}
      \item Important for any company that seeks to minimize cost of holding excess inventory while minimizing backorder costs (unmet demand)
      \item Particularly important in industries where the demand environment is highly dynamic. (i.e. Apple's supply-chain for solid state drives, seasonal products such as rock salt)
      \item These environments experience high correlation between demands in different periods (difficult to compute optimal inventory policy).
    \end{itemize}
  \end{frame}

  \section{Expected Cost Function}
  \subsection{Parameters}
  \begin{frame}
    \frametitle{Expected Cost Function and Parameters}
    \Large
    \[c_{t}\left(q_{t}\right)+E\left[h_{t}\left(x_{t}+q_{t}-D_{t}\right)^{+}+p_{t}\left(x_{t}+q_{t}-D_{t}\right)^{-}\right]\]
    \normalsize
    \begin{itemize}
      \item $h_{t}\left(x_{t}+q_{t}-D_{t}\right)^{+} \text{is holding cost}$
      \item $p_{t}\left(x_{t}+q_{t}-D_{t}\right)^{-} \text{is backlogging cost}$
      \item $x_{t}$ is inventory position (how much inventory you have in stock and on order)
      \item Cost periods $t,\cdots,T$
      \item $Q_{t}$ is the amount of product
      \item $D_{t}$ is the demand for the product (random)
      \item $c_{t}\left(q_{t}\right) \text{ is production cost}$
    \end{itemize}
  \end{frame}
  \subsection{Computational Challenges}
  \begin{frame}
  \frametitle{Computational Challenges}
  \vspace{0.5in}\par
    {\centering\yearwidth{0.8pt}\tclap{\tiny t}\showyear{1}\skipyears[black]{.3in}\showyear{2}\skipyears[black]{.3in}\showyear{3}\skipyears[black]{.3in}\showyear{4}\skipyears[black]{.3in}\showyear{5}\skipyears[black]{.3in}\showyear{6}\skipyears[black]{.3in}$\text{  }\cdots\text{  }$\showyear{$T-1$}\skipyears[black]{.3in}\showyear{$T$}
  \par}
  In the periodic-review stochastic inventory control problem, the cost consists of a per-unit, time-dependent ordering cost, a per-unit holding cost for carrying excess inventory from period to period, and a per-unit backlogging cost, which is a penalty we incur for each unit of unsatisfied demand (where all shortages are fully backlogged). In addition, there is a lead time between the time an order is placed and the time that it actually arrives

  \end{frame}

  \section{The Algorithm and Computational Approaches}
    \subsection{Dynamic Programming}
    \begin{frame}
    \frametitle{Dynamic Programming}
      \begin{eqnarray*}
      V_{t}\left(x_{t}\right) & = & \min_{q_{t}\geq0}c_{t}\left(q_{t}\right)+E\left[h_{t}\left(x_{t}+q_{t}-D_{t}\right)^{+}+p_{t}\left(x_{t}+q_{t}-D_{t}\right)^{-}\right]\\
       &  & + E\left[V_{t+1}\left(x_{t}+q_{t}-D_{t}\right)\right]
      \end{eqnarray*}
      \begin{itemize}
        \item Let $V_{t}\left(x_{t}\right)$ be the optimal expected cost over over periods $t=t,\cdots,T.$
        \item Using dynamic programming, we can solve each $V_{t}$
        \item This approach would be accurate if it were feasible, but the term $E\left[V_{t+1}\left(x_{t}+q_{t}-D_{t}\right)\right]$ is very difficult to compute.
      \end{itemize}
    \end{frame}

    \subsection{Myopic (Na{\"i}ve) Approach}
    \begin{frame}
    \frametitle{Myopic (Na{\"i}ve) Approach}
      The myopic approach focuses only on minimizing the expected immediate cost that is going to be incurred in each period, while ignoring the cost over the rest of the horizon.
    \end{frame}

    \subsection{Dual-Balancing Problem}
    \begin{frame}
    \frametitle{Dual-Balancing Problem}
      \[H_t^B(Q_t):=\sum\limits_{j=t+L}^T h_j(Q_t-(D_{[t,j]}-X_t)^+)^+\]
      \[\Pi_t^B:=p_t(D_{[t,t+L]}-(X_t^B+Q_t))^+=p_t(D_{[t,t+L]}-Y_t^B)^+\]
    \end{frame}

    \subsection{Why Dual-Balancing?}
    \begin{frame}
    \frametitle{Why Dual-Balancing?}

    \end{frame}
  \section{Coding and Implementation}
    \subsection{Coding and Implementation}
    \begin{frame}
    \frametitle{Coding and Implementation}

    \end{frame}
  \section{Results}
    \subsection{Time Complexity}
    \begin{frame}
    \frametitle{Time Complexity}
    Dual-balancing
    Myopic
    Dynamic Programming
    \end{frame}
    \subsection{Accuracy}
    \begin{frame}
    \frametitle{Accuracy}
    Dual-balancing vs. Myopic
    Dual-balancing vs. Dynamic Programming
    \end{frame}
  \section{Open Problems and Challenges in the Field of Stochastic Inventory Control}
    \subsection{Open Problems and Challenges in the Field of Stochastic Inventory Control}
    \begin{frame}
    \frametitle{Open Problems and Challenges in the Field of Stochastic Inventory Control}

    \end{frame}

\end{document}